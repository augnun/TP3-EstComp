\documentclass[a4paper,12pt,english,brazil]{article}
\usepackage[round,authoryear]{natbib}
\usepackage[brazilian]{babel}
\usepackage{amsfonts}
\usepackage{graphicx}
\usepackage[T1]{fontenc}
\usepackage{ae}
\usepackage[utf8]{inputenc}
\usepackage{amsmath}
\usepackage{url}
\usepackage{listings}
\lstset{
    language=R,
    basicstyle=\ttfamily
}

\begin{document}
\bibliographystyle{plainnat}


\title{Trabalho Prático 3 \\ Simulação de Sistema M$/$M$/$1}
\author{Grupo 13 \\ Alisson Moreira Ferreira - 11/0106946 \\ Augusto Cesar Ribeiro Nunes - 13/0103004}
\maketitle
\begin{abstract}
Este Trabalho Prático implementa e simula um Sistema M$/$M$/$1 na linguagem R. Utilizou-se $\lambda = 20$ como o parâmetro da Distribuição Poisson do número de clientes que entram no sistema a cada unidade de tempo, e $\mu = 50$ como parâmetro para Distribuição Poisson do número de atendimentos a cada unidade de tempo e disciplina de atendimento \textit{First In, First Out}. Foram obtidas estimativas médias e os Intervalos de Confiança para o Tempo Médio de Espera na fila (por cliente), para o Tamanho Médio da Fila e para a Taxa de Ocupação.
\end{abstract}

\section{Introdução}
A Teoria de Filas tem sua origem nos trabalhos do dinamarquês Agnes Krarup Erlang (\cite{wiki:001}) que entre 1917-20, enquanto trabalhava para a empresa de telefonia estatal da Dinamarca, foi incumbido da tarefa de otimizar a alocação de circuitos para que o serviço tivesse uma boa relação entre eficiência e custo de operação.

Este já era um problema conhecido: no final do século 19 e início do século 20 as redes de telefone, cujos circuitos eram manuseados por operadores, sofriam com problemas de dimensionamento causados pela súbita expansão no sistema. Esta expansão teve como causa fundamental o cancelamento judicial das patentes da AT\&T e Bell Company nos EUA: o país que até então tinha um monopólio do serviço se viu inundado por dezenas de novos provedores de serviço (\cite{northrup2011american}, \textit{Essay} "Communications"). 

Mas o problema que até então era tratado de maneira \textit{ad hoc}, foi sistematizado por Erlang em \textit{Telefon-Ventetider. Et Stykke Sandsynlighedsregning} - "Filas de Telefonia: Um pouco de Probabilidade" - o dinamarquês ilustrou a utilização de conceitos modernos de probabilidade na análise do problema em questão. O estudo de Erlang daria fruição ao ramo da Pesquisa Operacional que seria posteriormente conhecido como Engenharia de Teletráfico e à Teoria de Filas em Probabilidade e Estatística. 

Matematicamente, uma fila descreve a maneira como \textit{clientes} - quem ou o quê demanda um \textit{serviço} - são atendidos por \textit{servidores} - quem provê o \textit{serviço} em questão. O Modelo de Filas adotado aqui utiliza a Notação de Kendall (ver \citet{bose2013introduction}), que, com cinco parâmetros descreve a natureza do processo de chegada à fila, a natureza do processo de serviço (em termos de tempo de serviço), o número de \textit{servidores}, o número máximo de clientes no sistema e a disciplina de serviço básica. O Modelo M$/$M$/$1 implementado e simulado aqui utiliza os seguintes parâmetros:

\begin{itemize}
\item O processo de chegada dos clientes é o que se chama \textbf{Processo de Poisson}, ou seja,\textbf{ o número de clientes} que chegam ao sistema, a cada intervalo de tempo, \textbf{segue uma distribuição de Poisson} com taxa $\lambda = 20$. Equivalentemente, podemos afirmar que \textbf{o tempo entre chegadas consecutivas de clientes} segue uma \textbf{Distribuição Exponencial} com média $1/\lambda$\footnote{Daí a utilização de "M" na notação: diferentes bibliografias utilizam como indicação de que a distribuição é\textit{Markoviana} ou \textit{Memoryless}};
\item O \textbf{tempo de duração de cada atendimento} segue distribuição \textbf{Exponencial} com parâmetro $\mu = 50$;
\item O \textbf{número de servidores} disponíveis é igual a 1;
\item O \textbf{número de filas} também é igual a 1;
\item O \textbf{número máximo de clientes} é $\infty$;
\item A \textbf{Disciplina de Serviço} é \textit{First In, First Out}, ou seja, os primeiros clientes a chegarem são os primeiros a serem atendidos.
\end{itemize}

Ou seja, simulou-se uma fila onde o número de clientes que chegam a cada intervalo de tempo é, em média, igual a 20, e o tempo de duração de cada atendimento é, em média, igual a 50. Portanto, a \textbf{Taxa de Ocupação} do servidor é $\rho = \frac{\lambda}{\mu} = 2/5$, ou seja, o servidor fica ocupado, em média, 40\% do tempo de simulação. Equivalentemente, o servidor fica ocioso 60\% do tempo.

O Tempo Médio de Espera (por cliente) 

Um dos estágios essenciais para executar uma simulação é estipular o número de simulações que devem ser executadas. A técnica aqui utilizada (ver \cite{ross2013simulation}) para obter as estimativas médias do Tempo Médio de Espera \textbf{TME} na fila (por cliente), para o Tamanho Médio da Fila \textbf{TMF} e para a Taxa de Ocupação \textbf{TO}, resulta diretamente do Teorema do Limite Central: temos uma distribuição que atende às condições de regularidade exigidas, bem como média e variância finitas, as filas foram simuladas n vezes, admitindo um desvio padrão de $10^{-1}$ em todos os casos. Ou seja,

\begin{enumerate}
\item Foi escolhido $d = 10^{-1}$ como valor aceitável para o desvio-padrão da estimativa;
\item Foram simuladas inicialmente 100 filas;
\item Novas simulações foram obtidas\footnote{Notando que há um claro problema sob o ponto de vista teórico quando afirmamos algo do tipo, já que o tamanho da amostra passa a ser também aleatório}, até que obteve-se uma quantidade $k$ de simulações tais que $S/\sqrt{k} < d = 10^{-1}$
\item A estimativa para o parâmetro em questão foi então dada pela média das estimativas obtidas nas k simulações de forma a atender à tolerância estipulada.
\end{enumerate}
Então, estamos certos de que os valores simulados diferem dos valores "reais" em no máximo $1.96x10^{-1}$. Note que a escolha do desvio padrão aqui foi arbitrária, e em grande parte deveu-se a uma facilitação da obtenção das simulações com o parco poder computacional utilizado na confecção desse trabalho. Uma escolha mais criteriosa seria essencial em um contexto mais pragmático: as necessidades para a fila de um servidor \textit{web} ou para a fila de um caixa de lanchonete são grosseiramente diferentes, por exemplo, e além do fato de que a fila M$/$M$/$1 pode não ser a mais adequada, um desvio padrão de $10^{-1}$ tem implicações práticas extremamente dissonantes de acordo com o contexto. Portanto, como "prova de conceito", o valor escolhido \textit{funciona}, na "vida real", nem tanto.

Similarmente, para os Intervalos de Confiança de 95\% dos parâmetros citados acima, foram obtidos intervalos de confiança de amplitude L de no máximo $1$. Aqui vale um retorno sobre a observação feita para o desvio padrão das estimativas pontuais encontradas: novamente é um valor totalmente arbitrário e escolhido de certa forma a possibilitar a feitura deste relatório em tempo hábil, e em se tratando de problemas mais contextualizados dificilmente esta seria a escolha. Logo, podemos resumir nosso processo da seguinte forma:

\begin{enumerate}
\item Foram simuladas inicialmente 100 filas;
\item Novas simulações foram obtidas\footnote{Ver nota acima}, até que obteve-se uma quantidade $k$ de simulações tais que $2z_{\alpha/2}S/\sqrt{k} < L$. Os valores de S devem ser atualizados a cada nova simulação;
\item Se $\bar{x}$ e s são valores observados de $\bar{X}$ e S, então o Intervalo de Confiança de 95\% para o parâmetro em questão será dado por $\bar{x} \pm 1.96s/\sqrt{k}$
\end{enumerate}

    As implementações encontram-se no Anexo A. Foi disponibilizado um repositório no site github.com: \url{http://github.com/august-o/TP3-EstComp}. Lá encontram-se todos os arquivos-fonte utilizados neste Trabalho Prático. O computador utilizado foi um Macbook Pro, Sistema Operacional OS X El Capitan (10.11.3), com processador Intel Core i5 de 2.4 GHz e 16 GB de memória RAM 1333 MHz DDR3. A versão do R utilizada foi a 3.3.0, com a IDE RStudio em sua versão 0.99.1130. Como tentativa de mitigar a \textit{lentidão} do R que foi observada nos dois primeiros Trabalhos Práticos, optou-se por utilizar uma biblioteca de Álgebra Linear Básica (\textit{Basic Linear Algebra Subprograms} - \textbf{BLAS} em inglês) diferente da que vem como padrão no R: a vecLib\footnote{https://developer.apple.com/library/mac/documentation/Performance/Conceptual/vecLib/}.   

\section{Resultados}

As estimativas médias para os parâmetros e seus Intervalos de Confiança de 95\%, bem como o número de simulações necessárias para cada estimação, constam na Tabela \ref{tab:ests} a seguir:


\begin{table}[!htb]\centering
\begin{tabular}{ c |c| c}
Parâmetro & Estimativa & IC 95\% de confiança  \\ \hline
 TME & 5.087305 &  $5.087305 \pm 0.2256188$  \\  
 TMF & 0.007997178 & $0.007997178 \pm 0.00502039$ \\
 TO & 0.5940587 &  $0.5940587 \pm 1.197571e-05$\\

\end{tabular}
\caption{Estimativas pontuais e intervalares para os parâmetros considerando as tolerâncias discutidas}\label{tab:ests}
\end{table}

É importante retornar aqui à discussão tratada na introdução, quando foi explicada a metodologia de simulação. Todas as estimativas pontuais e intervalares acima acabaram atendendo aos critérios de parada já nas primeiras 100 simulações, então omitimos os números de simulações. A Tabela \ref{tab:aval} a seguir avalia como as estimativas pontuais da simulação aproximam os parâmetros:
\begin{table}[!htb]\centering
\begin{tabular}{ c |c| c|c|c}
Parâmetro & Estimativa & Parâmetro & Diferença ($\hat{\theta} - \theta)$ & Diferença Percentual \\ \hline
 TME & 5.087305 &  5.187 & -0.099695 & 1.9\% \\  
 TMF & 0.254673 & 0.2666667& -0.01199367& 4.5\% \\
 TO & 0.5940587 &  0.6 & -0.0059413 & 0.9\%\\

\end{tabular}
\caption{Avaliação das estimativas pontuais simuladas em relação aos parâmetros}\label{tab:aval}
\end{table}

\section{Conclusão}
Este Trabalho Prático teve algumas dificuldades de confecção: o código não foi devidamente otimizado e entendemos que um tempo de simulação maior (ou uma maior capacidade computacional) diminuiria as diferenças ilustradas na Tabela \ref{tab:aval}. No entanto, como passo inicial na simulação de um processo estocástico a tempo discreto o Trabalho Prático encontrou seu propósito, já que ilustrou importantes técnicas de simulação.
\newpage

\section*{Anexo A}
A seguir, os códigos e implementações utilizados no trabalho. Lembrando que os mesmos encontram-se em repositório no github: \url{https://github.com/august-o/TP3-EstComp}
\begin{lstlisting}
mm1.tme <- function(lambda = 20,
                    mu = 50,
                    tempo = 1000,
                    n=100) {
  tme <- vector()
  a <- 1
  while (a < n)
  {
    t <- 0
    i <- 1
    tc <- vector()
    ta <- vector()
    while (t < tempo) {
      tec <- rexp(1, rate = lambda)
      tc[i] <- t + tec
      t <- tc[i]
      i <- i + 1
    }
    to <- tc[1]
    te <- 0
    j <- 1
    t <- tc[1]
    while (j <= length(tc) - 1) {
      ta[j] <- rexp(1, rate = lambda) 
      if (t + ta[j] < tc[j + 1]) {
        to <- to + tc[j + 1] - (t + ta[j])
        t <- tc[j + 1]
      }
      else{
        te <- te + (t + ta[j]) - tc[j + 1]
        t <- t + ta[j]
      }
      j <- j + 1
    }
    tme[a] <- (te*length(tc))/ tempo
    a <- a + 1
  }
  return(tme)
}

mm1.ppo <- function(lambda = 20,
                    mu = 50,
                    tempo = 1000,
                    n=100) {
  ppo <- vector(length = n)
  a <- 1
  while (a < n)
  {
    t <- 0
    i <- 1
    tc <- vector()
    ta <- vector()
    while (t < tempo) {
      tec <- rexp(1, rate = lambda)
      tc[i] <- t + tec
      t <- tc[i]
      i <- i + 1
    }
    to <- tc[1]
    te <- 0
    j <- 1
    t <- tc[1]
    while (j <= length(tc) - 1) {
      ta[j] <- rexp(1, rate = mu) 
      if (t + ta[j] < tc[j + 1]) {
        to <- to + tc[j + 1] - (t + ta[j])
        t <- tc[j + 1]
      }
      else{
        te <- te + (t + ta[j]) - tc[j + 1]
        t <- t + ta[j]
      }
      j <- j + 1
    }
    ppo[a] <- to / tempo
    a <- a + 1
  }
  return(ppo)
}

mm1.tmf <- function(lambda = 20,
                    mu = 50,
                    tempo = 1000,
                    n=100) {
  tmf <- vector(length = n)
  a <- 1
  while (a <= n)
  {
    t <- 0
    i <- 1
    tc <- vector()
    ta <- vector()
    
    s <- 0
    while (t < tempo) {
      tec <- rexp(1, rate = lambda)
      tc[i] <- t + tec
      t <- tc[i]
      i <- i + 1
    }
    fila <- vector(length=length(tc))
    to <- tc[1]
    te <- 0
    j <- 1
    t <- tc[1]
    while (j <= length(tc) - 1) {
      ta[j] <- rexp(1, rate = mu) 
      if (t + ta[j] < tc[j + 1]) {
        to <- to + tc[j + 1] - (t + ta[j])
        t <- tc[j + 1]
      }
      else{
        te <- te + (t + ta[j]) - tc[j + 1]
        t <- t + ta[j]
        fila[j] <- fila[j] + 1
        s <- s + fila[j]*(tc[j+1]-tc[j])
        
      }

      j <- j + 1

    }
    tmf[a] <- s / length(tc)
    a <- a + 1
  }
  return(tmf)
}

set.seed(12345)
tme.vetor <- mm1.tme()
k <- 100
while(sd(tme.vetor)/k > 10^(-1)){
  append(tme.vetor, mm1.tme())
  k <- length(tme.vetor)
}
tme <- mean(tme.vetor)

while(2*pnorm(0.975)*sd(tme.vetor)/sqrt(k) > 1){
  append(tme.vetor, mm1.tme())
  k <- length(tme.vetor)
}
pnorm(0.975)*sd(tme.vetor)/sqrt(k)

set.seed(12345)
to.vetor <- mm1.ppo()
k <- 100
while(sd(to.vetor)/k > 10^(-1)){
  append(to.vetor, mm1.ppo())
  k <- length(to.vetor)
}
to <- mean(to.vetor)
to

while(2*pnorm(0.975)*sd(to.vetor)/sqrt(k) > 10^(-1)){
  append(to.vetor, mm1.ppo())
  k <- length(to.vetor)
}
pnorm(0.975)*sd(to.vetor)/sqrt(k)

set.seed(12345)
tmf.vetor <- mm1.tmf()
k <- 100
while(sd(tmf.vetor)/k > 10(-1)){
  append(tmf.vetor,mm1.tmf())
  k<-length(tmf.vetor)
}
tmf <- mean(tmf.vetor)
tmf

while(2*pnorm(0.975)*sd(tmf.vetor)/sqrt(k) > 10^(-1)){
  append(tmf.vetor, mm1.tmf())
  k <- length(tmf.vetor)
}
pnorm(0.975)*sd(tmf.vetor)/sqrt(k)


\end{lstlisting}



\bibliography{TP3-EstComp-Grupo13}

\end{document}